\documentclass{article}
\usepackage{caption} % Caption mynda
\usepackage{hyperref}
\usepackage{graphicx, float} % Insetning mynda
\usepackage{keystroke} % Veit ekki
\usepackage{pgfplots} % Veit ekki
\usepackage{enumerate} % Numbered-bullet-points
\usepackage{amsmath} % E.g. for formatting equations without numbers (equation*)
\usepackage{listings} % Insert code to document
\pgfplotsset{compat=1.17}
% \usepackage{showlabels}
\usepackage[T1]{fontenc} % Encoding
\usepackage[utf8]{inputenc} % Encoding
% \usepackage[icelandic]{babel} %Encoding

\usepackage{geometry}
\geometry{
    paper=a4paper, % letterpaper lika til
	top=2.5cm, % Top margin
	bottom=2cm, % Bottom margin
	left=3.5cm, % Left margin
	right=3.4cm, % Right margin
	headheight=0.75cm, % Header height
	footskip=1cm, % Space from the bottom margin to the baseline of the footer
	headsep=0.75cm, % Space from the top margin to the baseline of the header
	%showframe, % Uncomment to show how the type block is set on the page
}
    
% \usepgfplotslibrary{units}
\RequirePackage{fancyhdr}
\linespread{1.3} % Línubil
\fontfamily{pbch}\selectfont
\usepackage{color}

\definecolor{mygreen}{rgb}{0,0.6,0}
\definecolor{mygray}{rgb}{0.5,0.5,0.5}
\definecolor{mymauve}{rgb}{0.58,0,0.82}
\lstset{
showstringspaces=false,
backgroundcolor=\color{white},
basicstyle=\footnotesize,
breaklines=true,
captionpos=b,
commentstyle=\color{mygreen},
keywordstyle=\color{blue},
stringstyle=\color{mymauve},
}

% \usepackage{zref}
% \usepackage{biblatex} %Imports biblatex package
%\usepackage[backend=biber]{biblatex}
%\addbibresource{references.bib} %Import the bibliography file
\begin{document}

\today \par
\vspace{.5cm}
\noindent Háskólinn í Reykjavík, Embedded Systems Programming, \textbf{Project 1} \par
\noindent \textbf{Eyþór Mikael Eyþórsson}, \texttt{eythore19@ru.is}\par
%\noindent Instructors: \textbf{TEACHER} \par

\section*{Part 1}
To keep things convenient for engineering students, pulse rate was defined as the number
of positive edges per second, which has the unit $Hz$.
\subsection*{Max pulse rate}
The maximum pulse rate of the motor was found using the following formula: \[
    \text{max pulse rate} = \frac{\text{motor speed in rpm} \cdot \text{max pulses per
    revolution}}{60s}
\] which yielded \[
\text{max pulse rate} = \frac{155rpm \cdot (7\cdot 2\cdot 100)ppr}{60s} \approx 3617Hz
\]

\subsection*{Max time between samples}
The minimum sample rate was found using the Shannon Sampling Theorem, which
states that the sample rate must be at least twice the bandwidth of the signal to avoid
aliasing. Consequently, the maximum time between samples can be determined via \[
    \text{max time between samples} = \frac{1}{2*\text{sample rate in Hz}} =
    \frac{1}{2\cdot\frac{2.1\cdot 10^7ppm}{60s}} = 1.429\mu s
\]

\subsection*{Max response time}
To detect direction, a second encoder signal which is shifted by 90\textdegree{} is added
and so, to correctly determine the direction, the response time must be at maximum half of
the max time between samples, i.e. $\frac{1.429\mu s}{2} = 0.715\mu s$. The corresponding
sampling rate can be calculated as \[
    \text{sampling rate} = \frac{1}{1.429\mu s}Hz = 699.8kHz
\]

The response time was verified on the oscilloscope by adding delays to the input signal,
on each side of the threshold, verifying that delays under $0.715\mu s$ yielded a signal
proportionate to the rotation of the motor, while delays over $0.715\mu s$ yielded a
signal which were inversely proportional to the rotation of the motor.



%\newpage
%\printbibliography

\end{document}
